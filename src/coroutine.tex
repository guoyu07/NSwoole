\part{Coroutine}

\chapter{Overview}


Swoole 2.0增加了对协程(Coroutine)的支持,同时支持PHP5和PHP7,而且Swoole协程同时兼具了性能和并发能力。



\begin{lstlisting}[language=PHP]
<?php
$server = new Swoole\Http\Server('127.0.0.1',9501,SWOOLE_BASE);

$server->set(array('worker_num'=>1));

$server->on('Request',function($request,$response){
  $mysql = new Swoole\Coroutine\MySQL();
  $res=$mysql->connect([
      'host'=>'127.0.0.1',
      'user'=>'root',
      'password'=>'root',
      'database'=>'test'
   ]);
   if($res==false){
      $response->end('MySQL connect fail!');
      return;
   }
   $ret=$mysql->query('select sleep(1)');
   $response->end('swoole response is ok,result='.var_export($ret,true));
});
$server->start();
\end{lstlisting}

上述示例使用BASE模式创建一个Http服务器,并且设置进程为1,收到Http请求后执行一条select sleep(1) 语句,这条SQL模拟了阻塞IO,MySQL的服务器在1秒后返回结果。传统的同步阻塞程序,启动1个进程,毫无疑问只能提供1QPS的处理能力。而Swoole2.0协程不同,它虽然使用同步阻塞方式编写代码,但是底层确是异步IO,即使SQL语言执行时间很长,也可以提供很大的并发能力。


启动程序:

\begin{lstlisting}[language=PHP]
$ php server.php
$ ps aux|grep server.php
htf      13142  0.6  0.2 328480 42268 pts/24   S+   10:17   0:00 php server.php
\end{lstlisting}

并发100测试



\begin{lstlisting}[language=PHP]
$ ab -c 100 -n 1000 http://127.0.0.1:9501/
This is ApacheBench, Version 2.3 <$Revision: 1706008 $>
Copyright 1996 Adam Twiss, Zeus Technology Ltd, http://www.zeustech.net/
Licensed to The Apache Software Foundation, http://www.apache.org/

Benchmarking 127.0.0.1 (be patient)
Completed 100 requests
Completed 200 requests
Completed 300 requests
Completed 400 requests
Completed 500 requests
Completed 600 requests
Completed 700 requests
Completed 800 requests
Completed 900 requests
Completed 1000 requests
Finished 1000 requests


Server Software:        swoole-http-server
Server Hostname:        127.0.0.1
Server Port:            9501

Document Path:          /
Document Length:        85 bytes

Concurrency Level:      100
Time taken for tests:   10.061 seconds
Complete requests:      1000
Failed requests:        0
Total transferred:      233000 bytes
HTML transferred:       85000 bytes
Requests per second:    99.39 [#/sec] (mean)
Time per request:       1006.092 [ms] (mean)
Time per request:       10.061 [ms] (mean, across all concurrent requests)
Transfer rate:          22.62 [Kbytes/sec] received

Connection Times (ms)
              min  mean[+/-sd] median   max
Connect:        0    0   1.3      0       7
Processing:  1000 1004   4.7   1002    1020
Waiting:     1000 1004   4.7   1002    1020
Total:       1000 1005   5.8   1003    1026

Percentage of the requests served within a certain time (ms)
  50%   1003
  66%   1004
  75%   1005
  80%   1006
  90%   1016
  95%   1020
  98%   1022
  99%   1022
 100%   1026 (longest request)
\end{lstlisting}

并发1000测试



\begin{lstlisting}[language=PHP]
$ ab -c 1000 -n 2000 http://127.0.0.1:9501/
This is ApacheBench, Version 2.3 <$Revision: 1706008 $>
Copyright 1996 Adam Twiss, Zeus Technology Ltd, http://www.zeustech.net/
Licensed to The Apache Software Foundation, http://www.apache.org/

Benchmarking 127.0.0.1 (be patient)
Completed 200 requests
Completed 400 requests
Completed 600 requests
Completed 800 requests
Completed 1000 requests
Completed 1200 requests
Completed 1400 requests
Completed 1600 requests
Completed 1800 requests
Completed 2000 requests
Finished 2000 requests


Server Software:        swoole-http-server
Server Hostname:        127.0.0.1
Server Port:            9501

Document Path:          /
Document Length:        19 bytes

Concurrency Level:      1000
Time taken for tests:   2.025 seconds
Complete requests:      2000
Failed requests:        153
   (Connect: 0, Receive: 0, Length: 153, Exceptions: 0)
Total transferred:      344098 bytes
HTML transferred:       48098 bytes
Requests per second:    987.66 [#/sec] (mean)
Time per request:       1012.493 [ms] (mean)
Time per request:       1.012 [ms] (mean, across all concurrent requests)
Transfer rate:          165.94 [Kbytes/sec] received

Connection Times (ms)
              min  mean[+/-sd] median   max
Connect:        0    6   6.6      8      18
Processing:     8  166 336.4     68    2006
Waiting:        8  166 336.4     68    2006
Total:          8  173 340.6     83    2022

Percentage of the requests served within a certain time (ms)
  50%     83
  66%     87
  75%     89
  80%     90
  90%   1021
  95%   1087
  98%   1092
  99%   1093
 100%   2022 (longest request)
\end{lstlisting}

Length: 153 错误是因为MySQL服务器已经无法支持这么大并发了,拒绝了swoole的连接,因此会抛出错误。

为了进行对比,使用PHP7的PHP-FPM进行对比测试,修改php-fpm.conf将php-fpm进程设置为1。测试的代码与Swoole的完全一致,也是执行同一个sleep的SQL语句。 因为php-fpm是真正的同步阻塞,耗时太长,因此只能使用并发10请求100的方式测试。




\begin{lstlisting}[language=PHP]
<?php
$db = new mysqli;
$db->connect('127.0.0.1', 'root', 'root', 'test');
$result = $db->query("select sleep(1)");

if ($result) 
{
    print_r($result->fetch_all());
}
else die(sprintf("MySQLi Error: %s", mysqli_error($link)));
\end{lstlisting}

使用apache ab进行压力测试如下:


\begin{lstlisting}[language=PHP]
$ ab -c 10 -n 100 http://127.0.0.1/mysql.php
This is ApacheBench, Version 2.3 <$Revision: 1706008 $>
Copyright 1996 Adam Twiss, Zeus Technology Ltd, http://www.zeustech.net/
Licensed to The Apache Software Foundation, http://www.apache.org/

Benchmarking 127.0.0.1 (be patient).....done


Server Software:        nginx/1.10.0
Server Hostname:        127.0.0.1
Server Port:            80

Document Path:          /mysql.php
Document Length:        69 bytes

Concurrency Level:      10
Time taken for tests:   100.086 seconds
Complete requests:      100
Failed requests:        0
Total transferred:      24100 bytes
HTML transferred:       6900 bytes
Requests per second:    1.00 [#/sec] (mean)
Time per request:       10008.594 [ms] (mean)
Time per request:       1000.859 [ms] (mean, across all concurrent requests)
Transfer rate:          0.24 [Kbytes/sec] received

Connection Times (ms)
              min  mean[+/-sd] median   max
Connect:        0    0   0.1      0       0
Processing:  1002 9558 1636.0  10008   10011
Waiting:     1002 9558 1636.0  10008   10011
Total:       1002 9558 1636.0  10008   10011

Percentage of the requests served within a certain time (ms)
  50%  10008
  66%  10009
  75%  10009
  80%  10009
  90%  10009
  95%  10010
  98%  10010
  99%  10011
 100%  10011 (longest request)
\end{lstlisting}


Swoole服务器,即使SQL语句要执行1秒才返回结果,并发100的测试中也提供了100qps的能力,并发1000的测试中QPS为987,但MySQL服务器已经无法提供服务了。而同步阻塞的php-fpm处理并发10请求100的测试花费了100秒才完成,QPS为1。



PHP开发者可以基于Swoole2.0协程以同步的方式编写代码,底层自动进行协程调度,转变为异步IO,解决了传统异步编程嵌套回调的问题。

与Go语言协程不同,Swoole协程完全不需要开发者添加任何额外的关键词,直接以过去最传统的同步阻塞模式编写代码,底层自动进行协程调度实现异步IO,使并发编程变得非常简单。

与Node.js(ES6+)、Python等语言使用yield/generator、async/await的实现方式相比,Swoole协程无需修改代码添加额外的关键词。

与Go语言的goroutine相比,Swoole协程是内置式的,应用层代码无需添加额外的go关键词来启动协程,只需要使用封装好的协程客户端即可,使用更简单。



Swoole协程的IO组件在底层内置了超时机制,不需要使用复杂的select/chan/timer实现客户端超时。

Swoole底层内置的协程客户端组件包括:udpclient、tcpclient、httpclient、redisclient、mysqlclient,基本涵盖了开发者常用的几种通信协议。

协程组件只能在服务器的onConnect、onRequest、onReceive、onMessage 回调函数中使用。

\begin{lstlisting}[language=PHP]
<?php
$server = new Swoole\Http\Server('127.0.0.1',9501);

/*
 * 触发on request事件时,swoole会开辟一个协程栈,对协程栈进行初始化
 */
$server->on('Request',function($request,$response){
   $tcp_cli = new Swoole\Coroutine\Client(SWOOLE_SOCK_TCP);
   /*
    * client在调用connect函数后,swoole会将PHP上下文信息保存到当前栈内,
    * 然后将协程挂起,待确认连接成功后,触发epoll事件,然后协程切换,
    * 恢复PHP上下文信息,返回结果,继续执行PHP代码
    */
    if($tcp_cli->connect('127.0.0.1',9906)===false){
        $response->end('connect server failed'.);
        return;
    }
    $tcp_cli->send('test for the coroutine');
    /*
     * client在调用recv函数后,swoole会将PHP上下文信息保存到当前栈内
     * 然后将协程挂起待后端svr回包,触发epoll事件,然后协程切换,恢复
     * PHP上下文信息,返回结果,继续执行PHP代码。
     * 如果后端在设定的超时时间内,未能回包,返回false,client的errCode定为110
     */
     $ret = $tcp_cli->recv(100);
     $tcp_cli->close();
     if ($ret) {
         $response->end(' swoole response is ok');
     } else {
         $response->end(" recv failed error: {$tcp_cli->errCode}");
     }
});
$server->start();
\end{lstlisting}


\section{Swoole}


\subsection{UDP Client}


\begin{lstlisting}[language=PHP]
<?php
$udp_cli = new Swoole\Coroutine\Client(SWOOLE_SOCK_UDP);
$ret = $udp_cli->connect('127.0.0.1',9906);
$udp_cli->send('test for the coro');

$ret = $udp_cli->recv(100);
$udp_cli->close();

if($ret) {
   $response->end('swoole response is ok');
} else {
   $response->end("recv failed error: {$client->errCode}");
}
\end{lstlisting}

\subsection{MySQL Client}



\begin{lstlisting}[language=PHP]
<?php
$swoole_mysql = new Swoole\Coroutine\MySQL();
$swoole_mysql->connect([
   'host'=>'127.0.0.1',
   'user'=>'user',
   'password'=>'pass',
   'database'=>'test'
]);
$res=$swoole_mysql->query('select sleep(1)');
\end{lstlisting}

\subsection{Redis Client}


\begin{lstlisting}[language=PHP]
<?php
$redis = new Swoole\Coroutine\Redis();
$redis->connect('127.0.0.1',6379);
$val = $redis->get('key');
\end{lstlisting}

\subsection{HTTP Client}


\begin{lstlisting}[language=PHP]
<?php
$cli = new Swoole\Coroutine\Http\Client('127.0.0.1',80);
$cli->setHeader([
   'Host'=>'localhost',
   'User-Agent'=>'Chrome/49.0.2587.3',
   'Accept'=>'text/html,application/xhtml+xml,application/xml',
   'Accept-Encoding'=>'gzip'
]);
$cli->set(['timeout'=>1]);
$cli->get('/index.php');
echo $cli->body;
$cli->close();
\end{lstlisting}



\subsection{Link Pool}

Swoole2.0官方默认的实例是短连接的,在请求处理完毕后就会切断redis或mysql的连接。实际项目可以使用连接池实现复用。

实现原理也很简单,使用SplQueue,在请求到来时判断资源队列中是否有可用的连接,如果有直接拿来复用。如果没有就创建一个新的连接。在连接使用完毕后再讲它重新放回到队列,此连接就可以被其他协程复用。



\begin{lstlisting}[language=PHP]
<?php
$count = 0;
$pool = new SplQueue();
$server = new Swoole\Http\Server('127.0.0.1', 9501, SWOOLE_BASE);

$server->on('Request', function($request, $response) use(&$count, $pool) {
    if (count($pool) == 0) {
        $redis = new Swoole\Coroutine\Redis();
        $res = $redis->connect('127.0.0.1', 6379);
        if ($res == false) {
            $response->end("redis connect fail!");
            return;
        }
        $pool->push($redis);
        $count++;
    }
    $redis = $pool->pop();
    $ret = $redis->set('key', 'value');
    $response->end("swoole response is ok, count = $count, result=" . var_export($ret, true));
    $pool->push($redis);
});

$server->start();
\end{lstlisting}

此连接池代码也可用于其他协程客户端包括MySQL、HttpClient、TCPClient等。

协程没办法像异步回调程序一样可以对请求进行排队,然后通过事件驱动触发,没办法等到资源池可用。因此不太方便限制连接的数量。生产环境建议对连接池最大值做限制,操作最大连接数后,要拒绝客户端发起新的请求。


\begin{lstlisting}[language=PHP]
if (count($pool) == 0) {
    if ($count > 100) {
        $response->status(500);
        $response->end("<h1>连接池已满,无法提供服务,请稍后重试</h1>");
    }
}
\end{lstlisting}


\section{Fibjs}


FibJS 是一个建立在 Google v8 Javascript 引擎基础上的应用服务器开发框架,不同于 node.js,FibJS 采用 fiber 解决 v8 引擎的多路复用,并通过大量 c++ 组件,将重负荷运算委托给后台线程,释放 v8 线程,争取更大的并发时间。

相比JavaScriptCore和SpiderMonkey,使用v8引擎作为Fibjs基础核心的优势在于:

\begin{compactitem}
\item 支持多线程重入,协程的堆栈本质是线程,要支持协程必须要支持多堆栈重入;
\item 不支持多线程并发,虽然 isolate 支持多堆栈现场,但是 isolate 内部却为无锁环境,因此不能接受多线程同时运行,必须在一个线程退出后,另一线程才可进入。
\end{compactitem}


在v8引擎的基础上集成了协程环境,并且和v8进行交互以及其他基础模块就实现了Fibjs框架,因此Fibjs可以理解为是一个基于Google v8引擎实现的协程应用服务器开发框架。


\begin{example}
使用Fibjs实现HTTP服务器
\begin{lstlisting}[language=JavaScript]
var http = require('http');
var svr = new http.server(80,function(r){
   r.response.body.write(new Buffer('Hello,world'));
});
svr.run();
\end{lstlisting}
\end{example}

\begin{compactitem}
\item fibjs不是前端开发框架,不同于jQuery,Angular等运行在浏览器的JavaScript框架,fibjs运行在服务端。
\item fibjs不是Node.js的一个包,和NPM里面的fiber扩展包也没有关系。
\item fibjs是基于协程和V8,运用C++语言开发的JavaScript运行平台,和Node.js一样,都是服务端JavaScript环境。
\end{compactitem}

Fibjs支持使用同步语法来编写异步代码,虽然Nodejs可以通过Asyc,Promise,Generator等手段,在形式上简化回调写法,但是本质上没有变,始终无法靠直觉写出简洁优美的代码。



\begin{example}
Node.js回调实现文件异步读取
\begin{lstlisting}[language=JavaScript]
var fs = require('fs');

fs.readFile('file',function(err,data){
   if(err) throw err;
   console.log(data.toString());
});
\end{lstlisting}
\end{example}


\begin{example}
Fibjs实现文件异步读取
\begin{lstlisting}[language=JavaScript]
var fs = require('fs');
try {
   var file = fs.readFile('file');
   console.log(file);
} catch(e) {
   console.log(e.number);
}
\end{lstlisting}
\end{example}

fibjs的同步写法非常简洁,而且可以利用try catch来捕获异常,而node.js必须依赖回调来处理异步,而且需要采用Generator才能简化代码和错误处理。


\begin{example}
Nodejs基准测试
\begin{lstlisting}[language=JavaScript]
var http = require('http');
var server = http.createServer(function(request,response){
   response.writeHead(200, {
      "Content-Type":"text/html"
   });
   response.write("Hello world");
   response.end();
});

server.listen(8000);
\end{lstlisting}
\end{example}



\begin{lstlisting}[language=JavaScript]
$ wrk -t12 -c400 -d30s http://127.0.0.1:8000/
\end{lstlisting}


\begin{example}
Fibjs基准测试
\begin{lstlisting}[language=JavaScript]
var http = require('http');
var svr = new http.Server(8080,function(req){
   var rep = req.response;
   rep.addHeader({
      "Content-Type":"text/html"   
   });
   resp.body.write(new Buffer("Hello world"));
});

svr.run();
\end{lstlisting}
\end{example}



\begin{lstlisting}[language=JavaScript]
$ wrk -t12 -c400 -d30s http://127.0.0.1:8080/
\end{lstlisting}

Fibjs代码可以继续精简如下:

\begin{lstlisting}[language=JavaScript]
var http = require('http');
var svr = new http.Server(8080,function(req){
   req.response.body.write(new Buffer("Hello world"));
});

svr.run();
\end{lstlisting}

nodejs 追求最小核,大量基础是使用 js 实现,导致不能有效利用 js 引擎,fibjs 只是将 js 作为胶水语言,只在必要的时候才会切换至 js 引擎,使得 js 引擎更有效地被利用在应用逻辑,所以虽然 nodejs 和 fibjs 都是单线程 js + 多线程 worker thread 的工作模式,但是 fibjs 对 worker thread 的利用率比 nodejs 要大得多。


\begin{compactitem}
\item Fibjs使用全新的多线程 fiber 内核,更快,更省,更优雅;
\item Fibjs支持多 vm 并发,充分利用多核;
\item Fibjs的所有 async 接口支持 callback 调用模式;
\item Fibjs支持双向 pipe 通讯,支持超时,支持多进程;
\item Fibjs增加 global 变量,用于修改全局变量;
\item Fibjs增加 zip 模块,用于处理 zip 格式文件。
\end{compactitem}

\subsection{Windows}

Windows 下需要安装 Visual Studio 2015。


Windows 下点击start菜单选择所有程序选择Visual Studio 2015 选择Visual Studio Tools选择Developer Command Prompt for VS2015 打开终端后进入fibjs源代码目录,并执行命令:build




如果需要在Unix下编译Fibjs,需要预先安装的工具包括GCC、CMake、GNU Make和libexecinfo。


\begin{example}
在Mac OS X下准备Fibjs编译环境
\begin{lstlisting}[language=JavaScript]
$ brew install cmake git
\end{lstlisting}
\end{example}

\subsection{Ubuntu}


\begin{example}
在Ubuntu下准备Fibjs编译环境
\begin{lstlisting}[language=JavaScript]
$ sudo apt-get install g++ make cmake git
\end{lstlisting}
\end{example}


如果要编译 32 位版本,另需要安装 multilib:




\begin{lstlisting}[language=JavaScript]
$ sudo apt install gcc-multilib
$ sudo apt install g++-multilib
\end{lstlisting}


Arm on Ubuntu 准备环境命令如下:

\begin{lstlisting}[language=JavaScript]
apt install gcc-arm-linux-gnueabihf
apt install g++-arm-linux-gnueabihf
\end{lstlisting}



如果要在ubuntu上编译arm64位版本,准备环境命令如下:






\begin{lstlisting}[language=JavaScript]
apt install gcc-aarch64-linux-gnu
apt install g++-aarch64-linux-gnu
\end{lstlisting}


Mips on Ubuntu 准备环境如下:

\begin{lstlisting}[language=JavaScript]
apt install gcc-mips-linux-gnu
apt install g++-mips-linux-gnu
\end{lstlisting}


如果要在ubuntu上编译mips 64位版本,准备环境命令如下:


\begin{lstlisting}[language=JavaScript]
apt install gcc-mips64-linux-gnuabi64
apt install g++-mips64-linux-gnuabi64
\end{lstlisting}

Powerpc on Ubuntu 准备环境命令如下:


\begin{lstlisting}[language=JavaScript]
apt install gcc-powerpc-linux-gnu
apt install g++-powerpc-linux-gnu
\end{lstlisting}


如果要在ubuntu上编译powerpc 64位版本,准备环境命令如下:



\begin{lstlisting}[language=JavaScript]
apt install gcc-powerpc64-linux-gnu
apt install g++-powerpc64-linux-gnu
\end{lstlisting}




\begin{lstlisting}[language=JavaScript]
rm -f /usr/include/asm
ln -s x86_64-linux-gnu /usr/include/i386-linux-gnu
ln -s x86_64-linux-gnu /usr/include/x86_64-linux-gnux32
\end{lstlisting}



\subsection{Fedora}

Fedora 准备环境命令如下:

\begin{lstlisting}[language=JavaScript]
yum install gcc-c++
yum install libstdc++-static
yum install make
yum install cmake
yum install git
\end{lstlisting}

如果要编译 32 位版本,准备环境命令如下:


\begin{lstlisting}[language=JavaScript]
yum install glibc-devel.i686
yum install libstdc++-static.i686
\end{lstlisting}

\subsection{FreeBSD}


FreeBSD (8,9)准备环境命令如下:


\begin{lstlisting}[language=JavaScript]
pkg_add -r cmake
pkg_add -r libexecinfo
pkg_add -r git
\end{lstlisting}


FreeBSD 10及以上系统准备环境命令如下:

\begin{lstlisting}[language=JavaScript]
pkg install cmake
pkg install libexecinfo
pkg install git
\end{lstlisting}








获取代码并预处理:


\begin{lstlisting}[language=JavaScript]
git clone https://github.com/xicilion/fibjs.git
cd fibjs
git submodule init
git submodule update
\end{lstlisting}

在Unix 环境下编译Fibjs时,在 fibjs 项目根目录,有一个 build 的 shell 脚本,可用于 fibjs 编译。 

执行编译命令:


\begin{lstlisting}[language=JavaScript]
$ sh build [options] [-jn] [-v] [-h]
\end{lstlisting}

\begin{compactitem}
\item clean: 清除编译结果,便于全部重新编译
\item Release: 以发布方式编译
\item Debug: 以调试方式编译
\item i386: 以 32 位发布方式编译
\item amd64: 以 32 位发布方式编译
\item arm: 交叉编译32位arm版本
\item arm64: 交叉编译64位arm版本
\item mips: 交叉编译32位mips版本
\item mips64: 交叉编译64位mips版本
\item ppc: 交叉编译32位powerpc版本
\item ppc64: 交叉编译64位powerpc64版本
\end{compactitem}

例如:Release 模式编译命令如下:





\begin{lstlisting}[language=JavaScript]
$ sh build Release
\end{lstlisting}

进入 test 目录,执行 \texttt{../bin/\{\$OS\}\_Release/fibjs main.js} , 即可开始执行 fibjs 全部测试用例。


\begin{lstlisting}[language=JavaScript]
.......
db
  √ escape
  √ formatMySQL
  sqlite
    √ empty sql
    √ create table
    √ intert
    √ select
    √ callback
    √ binary (835ms) 
BUG:simple_api_call
  √ not hungup (109ms) 
√ 312 tests completed (6727ms)
\end{lstlisting}



到现在为止就得到了一个可以执行的 fibjs 版本了。

在 bin 目录下编辑生成一个 hello.js,内容如下:


\begin{lstlisting}[language=JavaScript]
console.log('Hello, World!');
\end{lstlisting}

然后即可执行:




\begin{lstlisting}[language=JavaScript]
$ Release/fibjs hello
\end{lstlisting}

可以看到屏幕输出:




\begin{lstlisting}[language=JavaScript]
Hello, World!
\end{lstlisting}

下面的时钟示例程序 dang.js,每隔一秒钟敲一声。当然也可以间隔时间更长。


\begin{lstlisting}[language=JavaScript]
var coroutine = require("coroutine");
while(true){
  console.log('DANG...');
  coroutine.sleep(1000);
}
\end{lstlisting}

执行后会看到屏幕中每隔一秒输出一个 \texttt{'DANG...'}

同一时刻只有一个时钟显然不是我们想要的,我们需要好多个时钟。


\begin{lstlisting}[language=JavaScript]
var coroutine = require('coroutine');

function dang(n){
   while(true){
      console.log('DANG %d...',n);
      coroutine.sleep(1000);
   }
}

for(var i = 0; i < 5; i ++) {
   coroutine.start(dang,i);
}

while(true){
   coroutine.sleep(1000);
}
\end{lstlisting}

我们于是拥有 5个时钟,每隔一秒同时提醒:


\begin{lstlisting}[language=JavaScript]
DANG 0...
DANG 1...
DANG 2...
DANG 3...
DANG 4...
...
DANG 0...
DANG 1...
DANG 2...
DANG 3...
DANG 4...
...
\end{lstlisting}



如果想更和谐一点,时钟们不要那么扎堆一起提醒,可以修改一下代码:


\begin{lstlisting}[language=JavaScript]
var coroutine = require('coroutine');

function dang(n) {
   while(true){
      console.log('DANG %d...',n);
      coroutine.sleep(1000);
   }
}

for(var i = 0; i < 5; i ++) {
   coroutine.start(dang,i);
   coroutine.sleep(200);
}

while(true){
   coroutine.sleep(1000);
}
\end{lstlisting}


\subsection{并行作业}


在很多时候,我们需要在一次请求完成多个独立的数据库操作,而且大部分时候,这些操作之间并无制约关系,通过协程就可以很方便地将这些操作平行发送,以提高响应速度。



\begin{lstlisting}[language=JavaScript]
var coroutine = require("coroutine");
function func(){
  coroutine.parallel(
    function(){
      //...
    }, 
    function(){
      //...
    }, 
    function(){
      //...
    }, 
    function(){
      //...
    }
  );
}
\end{lstlisting}

上述代码中的多个匿名函数会被同时执行,并在全部完成后,继续下面的代码。


\subsection{异步作业}


在每次处理用户请求时,总会有一些操作是不需要立即完成的,只有用户可以感知的部分需要在请求内完成,为了尽快响应用户,我们可以把用户暂时感知不到的请求丢到请求返回以后进行:

\begin{lstlisting}[language=JavaScript]
var coroutine = require("coroutine");
function func(){
  //...
  coroutine.start(function(){
    //...
  });
  //...
}
\end{lstlisting}

上述代码中匿名函数便会延迟执行,而用户得以迅速得到反馈。

和层层回调相比,协程可以以优雅的代码来编写并发代码。


\subsection{模块管理}


fibjs 采用 CommonJS 模块加载机制进行模块管理,支持直接使用同步和异步加载语法,以兼容前端代码。

模块加载由四个基本元素组成,包含两个函数 require() 函数, define() 函数 和两个对象 module 对象, exports 对象 。 


\subsubsection{require()}



\subsubsection{define()}


\subsubsection{module}



\subsubsection{exports}



\subsection{多核编程}


JavaScript 是个单线程编程语言,虽然 fibjs 通过 fiber 将并发引入 JavaScript 开发,但是并不能真正让 JavaScript 可以并行运算。

为了解决这个问题,fibjs 在 0.2.0 开始引入多 vm 运算,通过多个独立的 JavaScript vm 达到充分利用多核 cpu 的需求。




fibjs 启用多核运算的方式及其简单,因为多核运算的实现是基于多个 vm 的,而多个 vm 之间并不能共享数据,因此,在 vm 之间便需要通过 rpc 模块交互。示例如下:



\begin{lstlisting}[language=JavaScript]
var rpc = require("rpc");
var task = rpc.open("path/to.js");
console.log(task.func1(100, 200));
console.log(task.sub1.func2(100, 200, 300));
\end{lstlisting}

示例中,多核运算与普通模块引用的差别,主要是 rpc.open,通过 rpc.open,可以创建一个 RpcTask 代理对象,用于封装发往运算 vm 的请求和返回结果。


RpcTask 对象同时还负责对远程对象的子对象引用的封装,比如 task.sub1 就会创建一个新的 RpcTask 代理对象,指向 vm 中模块 path/to.js 的 sub1 对象。


因为多核运算需要通过 rpc 交互,所以在请求和返回时,便对数据有比较严格的约束。无论是传递参数,还是返回结果,都必须是可以序列化的数据。简单说就是必须是可以 json 编码解码的数据。如果传递一个有类型的对象,则只会传递为一个无类型的对象和数据来。


\subsection{服务器编程}


使用 fibjs 创建一个简单的 web 服务器是一件很方便的事情。




\begin{lstlisting}[language=JavaScript]
var http = require('http');
var svr = new http.Server(80, http.fileHandler('../fibjs/docs'));
svr.run();
\end{lstlisting}

这个 web 服务器已经足够强大,支持自动压缩,支持预压缩,支持更新检查,作为普通的静态文件服务已经足够。不过,apache 还是 nginx都是更专业的静态Web服务器,因此fibjs 的静态 web 服务器仅仅用于辅助完成一些不是很大量的前端文件下载。

动态Web服务器其实并不比静态文件服务器复杂更多,例如下面是一个简单的文本响应,如果需要构建更复杂的动态服务器则需要进一步修改。


\begin{lstlisting}[language=JavaScript]
var http = require('http');
var svr = new http.Server(80, function(r){
  r.response.write('Hello, World!');
});
svr.run();
\end{lstlisting}

实际上,当网站规模不大的时候,我们通常需要在同一台服务器上同时支持静态文件和动态请求:


\begin{lstlisting}[language=JavaScript]
var http = require('http');
var mq = require('mq');
var svr = new http.Server(8080, new mq.Routing({
    '^/www(/.*)': http.fileHandler('../fibjs/docs'),
    '^/rpc(/.*)': function(r) {
        r.response.write('Hello, World!');
    }
}));
svr.run();
\end{lstlisting}




\begin{lstlisting}[language=JavaScript]

\end{lstlisting}







\begin{lstlisting}[language=JavaScript]

\end{lstlisting}



\begin{lstlisting}[language=JavaScript]

\end{lstlisting}



\begin{lstlisting}[language=JavaScript]

\end{lstlisting}



\begin{lstlisting}[language=JavaScript]

\end{lstlisting}



\begin{lstlisting}[language=JavaScript]

\end{lstlisting}




\begin{lstlisting}[language=JavaScript]

\end{lstlisting}




\begin{lstlisting}[language=JavaScript]

\end{lstlisting}




\begin{lstlisting}[language=JavaScript]

\end{lstlisting}







\begin{lstlisting}[language=JavaScript]

\end{lstlisting}



\begin{lstlisting}[language=JavaScript]

\end{lstlisting}



\begin{lstlisting}[language=JavaScript]

\end{lstlisting}



\begin{lstlisting}[language=JavaScript]

\end{lstlisting}



\begin{lstlisting}[language=JavaScript]

\end{lstlisting}




\begin{lstlisting}[language=JavaScript]

\end{lstlisting}




\begin{lstlisting}[language=JavaScript]

\end{lstlisting}




\begin{lstlisting}[language=JavaScript]

\end{lstlisting}