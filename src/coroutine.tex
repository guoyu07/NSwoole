\part{Coroutine}

\chapter{Overview}


Swoole 2.0增加了对协程(Coroutine)的支持,同时支持PHP5和PHP7,而且Swoole协程同时兼具了性能和并发能力。



\begin{lstlisting}[language=PHP]
<?php
$server = new Swoole\Http\Server('127.0.0.1',9501,SWOOLE_BASE);

$server->set(array('worker_num'=>1));

$server->on('Request',function($request,$response){
  $mysql = new Swoole\Coroutine\MySQL();
  $res=$mysql->connect([
      'host'=>'127.0.0.1',
      'user'=>'root',
      'password'=>'root',
      'database'=>'test'
   ]);
   if($res==false){
      $response->end('MySQL connect fail!');
      return;
   }
   $ret=$mysql->query('select sleep(1)');
   $response->end('swoole response is ok,result='.var_export($ret,true));
});
$server->start();
\end{lstlisting}

上述示例使用BASE模式创建一个Http服务器,并且设置进程为1,收到Http请求后执行一条select sleep(1) 语句,这条SQL模拟了阻塞IO,MySQL的服务器在1秒后返回结果。传统的同步阻塞程序,启动1个进程,毫无疑问只能提供1QPS的处理能力。而Swoole2.0协程不同,它虽然使用同步阻塞方式编写代码,但是底层确是异步IO,即使SQL语言执行时间很长,也可以提供很大的并发能力。


启动程序:

\begin{lstlisting}[language=PHP]
$ php server.php
$ ps aux|grep server.php
htf      13142  0.6  0.2 328480 42268 pts/24   S+   10:17   0:00 php server.php
\end{lstlisting}

并发100测试



\begin{lstlisting}[language=PHP]
$ ab -c 100 -n 1000 http://127.0.0.1:9501/
This is ApacheBench, Version 2.3 <$Revision: 1706008 $>
Copyright 1996 Adam Twiss, Zeus Technology Ltd, http://www.zeustech.net/
Licensed to The Apache Software Foundation, http://www.apache.org/

Benchmarking 127.0.0.1 (be patient)
Completed 100 requests
Completed 200 requests
Completed 300 requests
Completed 400 requests
Completed 500 requests
Completed 600 requests
Completed 700 requests
Completed 800 requests
Completed 900 requests
Completed 1000 requests
Finished 1000 requests


Server Software:        swoole-http-server
Server Hostname:        127.0.0.1
Server Port:            9501

Document Path:          /
Document Length:        85 bytes

Concurrency Level:      100
Time taken for tests:   10.061 seconds
Complete requests:      1000
Failed requests:        0
Total transferred:      233000 bytes
HTML transferred:       85000 bytes
Requests per second:    99.39 [#/sec] (mean)
Time per request:       1006.092 [ms] (mean)
Time per request:       10.061 [ms] (mean, across all concurrent requests)
Transfer rate:          22.62 [Kbytes/sec] received

Connection Times (ms)
              min  mean[+/-sd] median   max
Connect:        0    0   1.3      0       7
Processing:  1000 1004   4.7   1002    1020
Waiting:     1000 1004   4.7   1002    1020
Total:       1000 1005   5.8   1003    1026

Percentage of the requests served within a certain time (ms)
  50%   1003
  66%   1004
  75%   1005
  80%   1006
  90%   1016
  95%   1020
  98%   1022
  99%   1022
 100%   1026 (longest request)
\end{lstlisting}

并发1000测试



\begin{lstlisting}[language=PHP]
$ ab -c 1000 -n 2000 http://127.0.0.1:9501/
This is ApacheBench, Version 2.3 <$Revision: 1706008 $>
Copyright 1996 Adam Twiss, Zeus Technology Ltd, http://www.zeustech.net/
Licensed to The Apache Software Foundation, http://www.apache.org/

Benchmarking 127.0.0.1 (be patient)
Completed 200 requests
Completed 400 requests
Completed 600 requests
Completed 800 requests
Completed 1000 requests
Completed 1200 requests
Completed 1400 requests
Completed 1600 requests
Completed 1800 requests
Completed 2000 requests
Finished 2000 requests


Server Software:        swoole-http-server
Server Hostname:        127.0.0.1
Server Port:            9501

Document Path:          /
Document Length:        19 bytes

Concurrency Level:      1000
Time taken for tests:   2.025 seconds
Complete requests:      2000
Failed requests:        153
   (Connect: 0, Receive: 0, Length: 153, Exceptions: 0)
Total transferred:      344098 bytes
HTML transferred:       48098 bytes
Requests per second:    987.66 [#/sec] (mean)
Time per request:       1012.493 [ms] (mean)
Time per request:       1.012 [ms] (mean, across all concurrent requests)
Transfer rate:          165.94 [Kbytes/sec] received

Connection Times (ms)
              min  mean[+/-sd] median   max
Connect:        0    6   6.6      8      18
Processing:     8  166 336.4     68    2006
Waiting:        8  166 336.4     68    2006
Total:          8  173 340.6     83    2022

Percentage of the requests served within a certain time (ms)
  50%     83
  66%     87
  75%     89
  80%     90
  90%   1021
  95%   1087
  98%   1092
  99%   1093
 100%   2022 (longest request)
\end{lstlisting}

Length: 153 错误是因为MySQL服务器已经无法支持这么大并发了,拒绝了swoole的连接,因此会抛出错误。

为了进行对比,使用PHP7的PHP-FPM进行对比测试,修改php-fpm.conf将php-fpm进程设置为1。测试的代码与Swoole的完全一致,也是执行同一个sleep的SQL语句。 因为php-fpm是真正的同步阻塞,耗时太长,因此只能使用并发10请求100的方式测试。




\begin{lstlisting}[language=PHP]
<?php
$db = new mysqli;
$db->connect('127.0.0.1', 'root', 'root', 'test');
$result = $db->query("select sleep(1)");

if ($result) 
{
    print_r($result->fetch_all());
}
else die(sprintf("MySQLi Error: %s", mysqli_error($link)));
\end{lstlisting}

使用apache ab进行压力测试如下:


\begin{lstlisting}[language=PHP]
$ ab -c 10 -n 100 http://127.0.0.1/mysql.php
This is ApacheBench, Version 2.3 <$Revision: 1706008 $>
Copyright 1996 Adam Twiss, Zeus Technology Ltd, http://www.zeustech.net/
Licensed to The Apache Software Foundation, http://www.apache.org/

Benchmarking 127.0.0.1 (be patient).....done


Server Software:        nginx/1.10.0
Server Hostname:        127.0.0.1
Server Port:            80

Document Path:          /mysql.php
Document Length:        69 bytes

Concurrency Level:      10
Time taken for tests:   100.086 seconds
Complete requests:      100
Failed requests:        0
Total transferred:      24100 bytes
HTML transferred:       6900 bytes
Requests per second:    1.00 [#/sec] (mean)
Time per request:       10008.594 [ms] (mean)
Time per request:       1000.859 [ms] (mean, across all concurrent requests)
Transfer rate:          0.24 [Kbytes/sec] received

Connection Times (ms)
              min  mean[+/-sd] median   max
Connect:        0    0   0.1      0       0
Processing:  1002 9558 1636.0  10008   10011
Waiting:     1002 9558 1636.0  10008   10011
Total:       1002 9558 1636.0  10008   10011

Percentage of the requests served within a certain time (ms)
  50%  10008
  66%  10009
  75%  10009
  80%  10009
  90%  10009
  95%  10010
  98%  10010
  99%  10011
 100%  10011 (longest request)
\end{lstlisting}


Swoole服务器,即使SQL语句要执行1秒才返回结果,并发100的测试中也提供了100qps的能力,并发1000的测试中QPS为987,但MySQL服务器已经无法提供服务了。而同步阻塞的php-fpm处理并发10请求100的测试花费了100秒才完成,QPS为1。



PHP开发者可以基于Swoole2.0协程以同步的方式编写代码,底层自动进行协程调度,转变为异步IO,解决了传统异步编程嵌套回调的问题。

与Go语言协程不同,Swoole协程完全不需要开发者添加任何额外的关键词,直接以过去最传统的同步阻塞模式编写代码,底层自动进行协程调度实现异步IO,使并发编程变得非常简单。

与Node.js(ES6+)、Python等语言使用yield/generator、async/await的实现方式相比,Swoole协程无需修改代码添加额外的关键词。

与Go语言的goroutine相比,Swoole协程是内置式的,应用层代码无需添加额外的go关键词来启动协程,只需要使用封装好的协程客户端即可,使用更简单。



Swoole协程的IO组件在底层内置了超时机制,不需要使用复杂的select/chan/timer实现客户端超时。

Swoole底层内置的协程客户端组件包括:udpclient、tcpclient、httpclient、redisclient、mysqlclient,基本涵盖了开发者常用的几种通信协议。

协程组件只能在服务器的onConnect、onRequest、onReceive、onMessage 回调函数中使用。

\begin{lstlisting}[language=PHP]
<?php
$server = new Swoole\Http\Server('127.0.0.1',9501);

/*
 * 触发on request事件时,swoole会开辟一个协程栈,对协程栈进行初始化
 */
$server->on('Request',function($request,$response){
   $tcp_cli = new Swoole\Coroutine\Client(SWOOLE_SOCK_TCP);
   /*
    * client在调用connect函数后,swoole会将PHP上下文信息保存到当前栈内,
    * 然后将协程挂起,待确认连接成功后,触发epoll事件,然后协程切换,
    * 恢复PHP上下文信息,返回结果,继续执行PHP代码
    */
    if($tcp_cli->connect('127.0.0.1',9906)===false){
        $response->end('connect server failed'.);
        return;
    }
    $tcp_cli->send('test for the coroutine');
    /*
     * client在调用recv函数后,swoole会将PHP上下文信息保存到当前栈内
     * 然后将协程挂起待后端svr回包,触发epoll事件,然后协程切换,恢复
     * PHP上下文信息,返回结果,继续执行PHP代码。
     * 如果后端在设定的超时时间内,未能回包,返回false,client的errCode定为110
     */
     $ret = $tcp_cli->recv(100);
     $tcp_cli->close();
     if ($ret) {
         $response->end(' swoole response is ok');
     } else {
         $response->end(" recv failed error: {$tcp_cli->errCode}");
     }
});
$server->start();
\end{lstlisting}

\section{UDP Client}


\begin{lstlisting}[language=PHP]
<?php
$udp_cli = new Swoole\Coroutine\Client(SWOOLE_SOCK_UDP);
$ret = $udp_cli->connect('127.0.0.1',9906);
$udp_cli->send('test for the coro');

$ret = $udp_cli->recv(100);
$udp_cli->close();

if($ret) {
   $response->end('swoole response is ok');
} else {
   $response->end("recv failed error: {$client->errCode}");
}
\end{lstlisting}

\section{MySQL Client}



\begin{lstlisting}[language=PHP]
<?php
$swoole_mysql = new Swoole\Coroutine\MySQL();
$swoole_mysql->connect([
   'host'=>'127.0.0.1',
   'user'=>'user',
   'password'=>'pass',
   'database'=>'test'
]);
$res=$swoole_mysql->query('select sleep(1)');
\end{lstlisting}

\section{Redis Client}


\begin{lstlisting}[language=PHP]
<?php
$redis = new Swoole\Coroutine\Redis();
$redis->connect('127.0.0.1',6379);
$val = $redis->get('key');
\end{lstlisting}

\section{HTTP Client}


\begin{lstlisting}[language=PHP]
<?php
$cli = new Swoole\Coroutine\Http\Client('127.0.0.1',80);
$cli->setHeader([
   'Host'=>'localhost',
   'User-Agent'=>'Chrome/49.0.2587.3',
   'Accept'=>'text/html,application/xhtml+xml,application/xml',
   'Accept-Encoding'=>'gzip'
]);
$cli->set(['timeout'=>1]);
$cli->get('/index.php');
echo $cli->body;
$cli->close();
\end{lstlisting}



\section{Link Pool}

Swoole2.0官方默认的实例是短连接的,在请求处理完毕后就会切断redis或mysql的连接。实际项目可以使用连接池实现复用。

实现原理也很简单,使用SplQueue,在请求到来时判断资源队列中是否有可用的连接,如果有直接拿来复用。如果没有就创建一个新的连接。在连接使用完毕后再讲它重新放回到队列,此连接就可以被其他协程复用。



\begin{lstlisting}[language=PHP]
<?php
$count = 0;
$pool = new SplQueue();
$server = new Swoole\Http\Server('127.0.0.1', 9501, SWOOLE_BASE);

$server->on('Request', function($request, $response) use(&$count, $pool) {
    if (count($pool) == 0) {
        $redis = new Swoole\Coroutine\Redis();
        $res = $redis->connect('127.0.0.1', 6379);
        if ($res == false) {
            $response->end("redis connect fail!");
            return;
        }
        $pool->push($redis);
        $count++;
    }
    $redis = $pool->pop();
    $ret = $redis->set('key', 'value');
    $response->end("swoole response is ok, count = $count, result=" . var_export($ret, true));
    $pool->push($redis);
});

$server->start();
\end{lstlisting}

此连接池代码也可用于其他协程客户端包括MySQL、HttpClient、TCPClient等。

协程没办法像异步回调程序一样可以对请求进行排队,然后通过事件驱动触发,没办法等到资源池可用。因此不太方便限制连接的数量。生产环境建议对连接池最大值做限制,操作最大连接数后,要拒绝客户端发起新的请求。


\begin{lstlisting}[language=PHP]
if (count($pool) == 0) {
    if ($count > 100) {
        $response->status(500);
        $response->end("<h1>连接池已满,无法提供服务,请稍后重试</h1>");
    }
}
\end{lstlisting}




\begin{lstlisting}[language=PHP]

\end{lstlisting}




\begin{lstlisting}[language=PHP]

\end{lstlisting}