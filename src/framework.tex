\part{Framework}


\chapter{Overview}


与其他Web框架不同,SwooleFramework是一个全功能的后端服务器框架。除了Web方面的应用之外,更广泛的后端程序中都可以使用。


\begin{compactitem}
\item 内置PHP应用服务器,可脱离nginx/php-fpm/apache独立运行
\item 配置化与资源自动工厂,可实现从配置中创建资源对象,完全无需new对象
\item 全面采用命名空间+autoload,代码中无需任何的include/require
\item 全局注册树,所有资源都挂载到全局树上,彻底实现资源的单例管理和懒加载
\item 全栈框架,提供了数据库操作,模板,Cache,日志,队列,上传管理,用户管理等几乎所有的功能
\end{compactitem}

使用内置应用服务器,可节省每次请求代码来的额外消耗,需要安装swoole扩展才能使用SwooleFramework应用服务器。

\begin{compactenum}
\item 使用PECL或手动编译安装Swoole扩展

\begin{lstlisting}[language=bash]
$ sudo pecl install swoole
\end{lstlisting}

\item 修改php.ini加入extension=swoole.so
\end{compactenum}

连接池技术可以很好的帮助存储系统节省连接资源。

Swoole应用服务器支持的特性如下:

\begin{compactitem}
\item 热部署,代码更新后即刻生效,依赖runkit扩展实现。
\item MaxRequest进程回收机制,防止内存泄露
\item 支持使用Windows作为开发环境
\item http KeepAlive,可节省tcp connect带来的开销
\item 静态文件缓存,节省流量
\item 支持Gzip压缩,节省流量
\item 支持MySQL重新连接
\item 支持文件上传
\item 支持POST大文本
\item 支持Session/Cookie
\item 支持Http/FastCGI两种协议
\end{compactitem}

Swoole框架额外提供的网络协议包括WebSocket、HTTP、SOA、SMTP、FTP和异步HTTPClient。

\begin{compactitem}
\item WebSocket协议支持,并附带一个基于websocket协议的webim系统
\item 普通Web服务器,可支持静态文件和普通include php方式的程序
\item SOA逻辑层服务器/客户端,支持并行请求
\item 一个简单的SMTP服务器
\item FtpServer
\item 异步HttpClient
\end{compactitem}



\begin{lstlisting}[language=bash]
<?php
require __DIR__.'/libs/lib_config.php';

$AppSvr = new Swoole\Protocol\AppServer();
$AppSvr->loadSetting(__DIR__."/swoole.ini"); // 加载配置文件
$AppSvr->setAppPath(__DIR__.'/apps/'); // 设置应用所在的目录
$AppSvr->setLogger(new Swoole\Log\EchoLog(false)); // Logger

/**
  * 如果没有安装swoole扩展,这里还可选择
  * 1. BlockTCP 阻塞的TCP,支持windows平台,需要将worker_num设为1
  * 2. SelectTCP 使用select做事件循环,支持windows平台,需要将worker_num设为1
  * 3. EventTCP 使用libevent,需要安装libevent扩展
  */
$server = new \Swoole\Network\Server('0.0.0.0',9501);
$server->setProtocol($AppSvr);
$server->daemonize($AppSvr);
$server->run(array('work_num'=>1,'max_request'=>5000));
\end{lstlisting}

在浏览器中打开\url{http://127.0.0.1:9501/}访问并测试Swoole应用程序。

\begin{lstlisting}[language=bash]
$ sudo php server.php
[2013-07-09 12:17:05]  Swoole. running. on 0.0.0.0:9501
\end{lstlisting}


下面的测试是使用swoole扩展作为底层Server框架的,其他驱动暂未测试。



\begin{lstlisting}
ab -c 100 -n 100000 http://127.0.0.1:9501/hello/index/
This is ApacheBench, Version 2.3 <$Revision: 655654 $>
Copyright 1996 Adam Twiss, Zeus Technology Ltd, http://www.zeustech.net/
Licensed to The Apache Software Foundation, http://www.apache.org/

Benchmarking 127.0.0.1 (be patient)
Completed 10000 requests
Completed 20000 requests
Completed 30000 requests
Completed 40000 requests
Completed 50000 requests
Completed 60000 requests
Completed 70000 requests
Completed 80000 requests
Completed 90000 requests
Completed 100000 requests
Finished 100000 requests


Server Software:        Swoole
Server Hostname:        127.0.0.1
Server Port:            9501

Document Path:          /hello/index/
Document Length:        11 bytes

Concurrency Level:      100
Time taken for tests:   10.717 seconds
Complete requests:      100000
Failed requests:        0
Write errors:           0
Total transferred:      27500000 bytes
HTML transferred:       1100000 bytes
Requests per second:    9330.83 [#/sec] (mean)
Time per request:       10.717 [ms] (mean)
Time per request:       0.107 [ms] (mean, across all concurrent requests)
Transfer rate:          2505.84 [Kbytes/sec] received

Connection Times (ms)
              min  mean[+/-sd] median   max
Connect:        0    1   1.0      1       9
Processing:     1   10   5.6      8      63
Waiting:        0    7   5.4      6      62
Total:          1   11   5.5      9      63

Percentage of the requests served within a certain time (ms)
  50%      9
  66%     11
  75%     12
  80%     13
  90%     17
  95%     22
  98%     28
  99%     32
 100%     63 (longest request)
\end{lstlisting}





\section{URLRewrite}


\subsection{Nginx+FPM+Swoole}



\begin{lstlisting}[language=bash]
server {
  listen 80;
  server_name www.swoole.com;
  root /data/wwwroot/www.swoole.com;
  
  location / {
    if(!-e $request_filename){
       proxy_pass http://127.0.0.1:9501;
    }
  }
}
\end{lstlisting}


\subsection{Apache+Swoole}



\begin{lstlisting}[language=bash]
<VirtualHost *:80>
    ServerName www.swoole.com
    DocumentRoot /data/webroot/www.swoole.com
    DirectoryIndex index.html index.php

    <Directory "/data/webroot/www.swoole.com">
        Options Indexes FollowSymLinks
            AllowOverride None
            Require all granted
    </Directory>

#   ProxyPass /admin !
#   ProxyPass /index.html !
#   ProxyPass /static !
#   ProxyPass / http://127.0.0.1:9501/

    <IfModule mod_rewrite.c>
        RewriteEngine On
        RewriteCond %{DOCUMENT_ROOT}/%{REQUEST_FILENAME} !-f
        RewriteCond %{DOCUMENT_ROOT}/%{REQUEST_FILENAME} !-d
        RewriteRule ^(.*)$ http://127.0.0.1:9501$1 [L,P]
    </IfModule>
</VirtualHost>
\end{lstlisting}










\begin{lstlisting}[language=bash]

\end{lstlisting}






\begin{lstlisting}[language=bash]

\end{lstlisting}






\begin{lstlisting}[language=bash]

\end{lstlisting}





\begin{lstlisting}[language=bash]

\end{lstlisting}





\begin{lstlisting}[language=bash]

\end{lstlisting}





\begin{lstlisting}[language=bash]

\end{lstlisting}





\begin{lstlisting}[language=bash]

\end{lstlisting}






\begin{lstlisting}[language=bash]

\end{lstlisting}






\begin{lstlisting}[language=bash]

\end{lstlisting}





\begin{lstlisting}[language=bash]

\end{lstlisting}





\begin{lstlisting}[language=bash]

\end{lstlisting}





\begin{lstlisting}[language=bash]

\end{lstlisting}





\begin{lstlisting}[language=bash]

\end{lstlisting}






\begin{lstlisting}[language=bash]

\end{lstlisting}






\begin{lstlisting}[language=bash]

\end{lstlisting}





\begin{lstlisting}[language=bash]

\end{lstlisting}





\begin{lstlisting}[language=bash]

\end{lstlisting}





\begin{lstlisting}[language=bash]

\end{lstlisting}





\begin{lstlisting}[language=bash]

\end{lstlisting}






\begin{lstlisting}[language=bash]

\end{lstlisting}
