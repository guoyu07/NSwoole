\part{Swoole}



\chapter{Introduction}


Swoole是一种基于PHP核心开发的高性能网络通信框架\footnote{Swoole本质是PHP的一种异步并行扩展,因此要应用Swoole,首先需要有PHP环境。},提供了PHP语言的异步多线程服务器,异步TCP/UDP网络客户端,异步MySQL,数据库连接池,AsyncTask,消息队列,毫秒定时器,异步文件读写,异步DNS查询。

\begin{compactitem}
\item swoole\_server,高并发高性能功能强大的异步并行TCP/UDP Server。
\item swoole\_client,支持同步/异步/并发的socket客户端实现。
\item swoole\_event,基于epoll/kqueue的全自动IO事件发生器。
\item swoole\_task,基于进程池实现的异步任务处理器。
\end{compactitem}

其中,swoole\_event比libevent更简单,仅需add/set/del几个操作即可,这样使用者就可以将原有PHP代码中的streams/fsockopen/sockets代码加入到swoole实现异步化,而且利用swoole\_event还可以实现真正的PHP异步MySQL。

用户可以使用swoole\_task实现PHP的数据库连接池,慢操作异步化,可以说Swoole开始将多线程、异步、阻塞引入PHP应用开发。

实际上,Swoole底层内置了异步非阻塞、多线程的网络IO服务器,这样仅需处理事件回调即可,无需关心底层。

与Nginx/Tornado/Node.js等全异步的框架不同,Swoole既支持全异步,也支持同步。

\section{Installation}

在下载Swoole源码包后,解压至本地任意目录(保证读写权限),而且安装PHP前,需要安装编译环境和PHP的相关依赖。

\begin{lstlisting}[language=bash]
$ sudo apt-get install \
	build-essential  \
	gcc \
	g++ \
	autoconf \
	libiconv-hook-dev \
	libmcrypt-dev \
	libxml2-dev \
	libmysqlclient-dev \
	libcurl4-openssl-dev \
	libjpeg8-dev \
	libpng12-dev \
	libfreetype6-dev
\end{lstlisting}

如果在CentOS/RHEL环境下编译安装PHP,则需要预先执行:

\begin{lstlisting}[language=bash]
$ sudo yum -y install \
	gcc \
	gcc-c++ \
	autoconf \
	libjpeg \
	libjpeg-devel \
	libpng \
	libpng-devel \
	freetype \
	freetype-devel \
	libxml2 \
	libxml2-devel \
	zlib \
	zlib-devel \
	glibc \
	glibc-devel \
	glib2 \
	glib2-devel \
	bzip2 \
	bzip2-devel \
	ncurses \
	ncurses-devel \
	curl \
	curl-devel \
	e2fsprogs \
	e2fsprogs-devel \
	krb5 \
	krb5-devel \
	libidn \
	libidn-devel \
	openssl \
	openssl-devel \
	openldap \
	openldap-devel \
	nss_ldap \
	openldap-clients \
	openldap-servers \
	gd \
	gd2 \
	gd-devel \
	gd2-devel \
	perl-CPAN \
	pcre-devel
$ cd php-src
$ ./configure \
	--prefix=/usr/local/php \
	--with-config-file-path=/etc/php \
	--enable-fpm \
	--enable-pcntl \
	--enable-mysqlnd \
	--enable-opcache \
	--enable-sockets \
	--enable-sysvmsg \
	--enable-sysvsem \
	--enable-sysvshm \
	--enable-shmop \
	--enable-zip \
	--enable-ftp \
	--enable-soap \
	--enable-xml \
	--enable-mbstring \
	--disable-rpath \
	--disable-debug \
	--disable-fileinfo \
	--with-mysql=mysqlnd \
	--with-mysqli=mysqlnd \
	--with-pdo-mysql=mysqlnd \
	--with-pcre-regex \
	--with-iconv \
	--with-zlib \
	--with-mcrypt \
	--with-gd \
	--with-openssl \
	--with-mhash \
	--with-xmlrpc \
	--with-curl \
	--with-imap-ssl
$ sudo make
$ sudo make install
$ sudo cp php.ini-development /etc/php/
$ cat >> ~/.bashrc
export PATH=/usr/local/php/bin:$PATH
export PATH=/usr/local/php/sbin:$PATH

$ source ~/.bashrc
$ php --version
PHP 5.6.12 (cli) (built: Aug 31 2015 11:09:49) 
Copyright (c) 1997-2015 The PHP Group
Zend Engine v2.6.0, Copyright (c) 1998-2015 Zend Technologies
$ cd /usr/local/src/
$ sudo tar zxvf swoole-1.7.19-stable.tar.gz
$ cd swoole-src-swoole-1.7.19-stable
$ sudo phpize
$ sudo ./configure \
	--enable-async-mysql \
	--enable-openssl
$ sudo make
$ sudo make install
$ sudo echo "extension=swoole.so" >>  /usr/local/lib/php.ini
$ php -m
[PHP Modules]
Core
ctype
date
dom
ereg
filter
hash
iconv
json
libxml
mysql
mysqlnd
pcre
PDO
pdo_sqlite
Phar
posix
propro
raphf
Reflection
session
SimpleXML
SPL
sqlite3
standard
swoole
tokenizer
xml
xmlreader
xmlwriter

[Zend Modules]
\end{lstlisting}







\begin{lstlisting}[language=bash]

\end{lstlisting}



\begin{lstlisting}[language=bash]

\end{lstlisting}



\begin{lstlisting}[language=bash]

\end{lstlisting}



\begin{lstlisting}[language=bash]

\end{lstlisting}



\begin{lstlisting}[language=bash]

\end{lstlisting}


\begin{lstlisting}[language=bash]

\end{lstlisting}

\begin{lstlisting}[language=bash]

\end{lstlisting}



\begin{lstlisting}[language=bash]

\end{lstlisting}




\begin{lstlisting}[language=bash]

\end{lstlisting}

